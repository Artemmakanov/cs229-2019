\item \points{10} {\bf PCA} 

In class, we showed that PCA finds the ``variance maximizing'' directions onto
which to project the data.  In this problem, we find another interpretation of PCA. 

Suppose we are given a set of points $\{x^{(1)},\ldots,x^{(\nexp)}\}$. Let us
assume that we have as usual preprocessed the data to have zero-mean and unit variance
in each coordinate.  For a given unit-length vector $u$, let $f_u(x)$ be the 
projection of point $x$ onto the direction given by $u$.  I.e., if 
${\cal V} = \{\alpha u : \alpha \in \Re\}$, then 
\[
f_u(x) = \arg \min_{v\in {\cal V}} ||x-v||^2.
\]
Show that the unit-length vector $u$ that minimizes the 
mean squared error between projected points and original points corresponds
to the first principal component for the data. I.e., show that
$$ \arg \min_{u:u^Tu=1} \sum_{i=1}^\nexp \|x^{(i)}-f_u(x^{(i)})\|_2^2 \ .$$
gives the first principal component.


{\bf Remark.} If we are asked to find a $k$-dimensional subspace onto which to
project the data so as to minimize the sum of squares distance between the
original data and their projections, then we should choose the $k$-dimensional
subspace spanned by the first $k$ principal components of the data.  This problem
shows that this result holds for the case of $k=1$.

\ifnum\solutions=1 {
  \begin{answer}
\begin{enumerate}

\item
\subitem  (Symm)$K(x,z) = K_1(x,z) + K_2(x,z) = K_1(z,x) + K_2(z,x) = K(z,x)$
\subitem (PSD) $x^TK_1x + x^TK_2x = x^T(K_1 + K_2)x \ge 0$ 
\item (not PSD) $
x^T
\begin{bmatrix}
    1  & 0 \\
    0  & 1
\end{bmatrix} x - 
x^T
\begin{bmatrix}
    2  & 0 \\
    0  & 2
\end{bmatrix} x =
x^T
\begin{bmatrix}
    -1  & 0 \\
    0  & -1
\end{bmatrix} x =
- (x_1^2 + x_2^2) \le 0
$
\item 
\subitem  (Symm)$K(x,z) = a K_1(x,z) = a K_1(z,x) = K(z,x)$
\subitem (PSD) $x^TK(x,z)x = a x^TK_1(x,z)x \ge 0$
\item (not PSD) 
$
\begin{bmatrix}
    1  & 0 \\
    0  & 1
\end{bmatrix} 
$ - kernel
$
\begin{bmatrix}
    -1  & 0 \\
    0  & -1
\end{bmatrix} 
$ - not a kernel
\item
\subitem  (Symm)$K(x,z) = K_1(x,z)K_2(x,z) =  K_1(z,x)K_2(z,x)= K(z,x)$
\subitem (PSD) $K(x^{(i)}, x^{(j)}) =K_1(x^{(i)}, x^{(j)})K_2(x^{(i)}, x^{(j)}) = (\sum_{p, q} \phi_1^{(p)}(x^{(i)}) \phi_1^{(q)}(x^{(j)}))  (\sum_{p, q} \phi_2^{(p)}(x^{(i)}) \phi_2^{(q)}(x^{(j)})) =$
$\sum_{p, q} \phi_1^{(p)}(x^{(i)}) \phi_1^{(q)}(x^{(i)}) \phi_2^{(p)}(x^{(i)}) \phi_2^{(q)}(x^{(i)})$ 
$x^TKx = \sum_{p, q}  \sum_{i, j} x^{(i)} \phi_1^{(p)}(x^{(i)}) \phi_1^{(q)}(x^{(i)}) \phi_2^{(p)}(x^{(i)}) \phi_2^{(q)}(x^{(i)}) x^{(j)} =$
$\sum_{p, q} (\sum_{i} x^{(i)} \phi_1^{(p)}(x^{(i)}) \phi_2^{(q)}(x^{(i)}))^2$
\item 
\subitem  (Symm)$K(x,z) = f(x) f(z) = f(z) f(x) = K(z,x)$
\subitem (PSD) $x^TKx = \sum_{i, j} x^{(i)} f(x^{(i)}) x^{(j)} f(x^{(j)}) = (\sum_{i}x^{(i)} f(x^{(i)}) )^2 \ge 0$
\item
\subitem  (Symm) $K(x,z) = K_3(\phi(x), \phi(z)) = K_3(\phi(z), \phi(x)) = K(z,x)$
\subitem (PSD) $x^TKx = \sum_{i, j} x^{(i)} f(\phi(x^{(i)})) x^{(j)} f(\phi(x^{(j)})) = (\sum_{i}x^{(i)} f(\phi(x^{(i)})) )^2 \ge 0$
\item 
\subitem  (Symm) $K(x,z) = p(K_1(x, z)) = p(K_1(z, x))= K(z,x)$
\subitem (PSD) $
K = 
a_0\begin{bmatrix}
    1 & \dots  & 1 \\
    \vdots & \ddots & \vdots \\
    1 & \dots  & 1
\end{bmatrix} + 
a_1\begin{bmatrix}
    K_1(x^{(1)},x^{(1)}) & \dots  & K_1(x^{(1)},x^{(n)})  \\
    \vdots & \ddots & \vdots \\
    K_1(x^{(n)},x^{(1)})  & \dots  & K_1(x^{(n)},x^{(n()}) 
\end{bmatrix} + $ \\
$
a_2\begin{bmatrix}
    K_1(x^{(1)},x^{(1)})^2 & \dots  & K_1(x^{(1)},x^{(n)})^2  \\
    \vdots & \ddots & \vdots \\
    K_1(x^{(n)},x^{(1)})^2  & \dots  & K_1(x^{(n)},x^{(n()})^2
\end{bmatrix}   + ...
$
\subitem 0. $x^Ta_0\begin{bmatrix}
    1 & \dots  & 1 \\
    \vdots & \ddots & \vdots \\
    1 & \dots  & 1
\end{bmatrix}x \ge 0 $
\subitem 1. $a_1 x^T K_1 x \ge 0$
\subitem 2. $a_1 x^T K_1 \circ K_1 x \ge 0$ (consequence from (e))
\subitem 3. ...
\end{enumerate}

\end{answer}

} \fi

  
