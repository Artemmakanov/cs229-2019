\item \points{18} {\bf Constructing kernels}

In class, we saw that by choosing a kernel $K(x,z) = \phi(x)^T\phi(z)$, we can
implicitly map data to a high dimensional space, and have a learning algorithm (e.g SVM or logistic regression)
work in that space. One way to generate kernels is to explicitly define the
mapping $\phi$ to a higher dimensional space, and then work out the
corresponding $K$.

However in this question we are interested in direct construction of kernels.
I.e., suppose we have a function $K(x,z)$ that we think gives an appropriate
similarity measure for our learning problem, and we are considering plugging
$K$ into the SVM as the kernel function. However for $K(x,z)$ to be a valid
kernel, it must correspond to an inner product in some higher dimensional space
resulting from some feature mapping $\phi$.  Mercer's theorem tells us that
$K(x,z)$ is a (Mercer) kernel if and only if for any finite set $\{x^{(1)},
\ldots, x^{(\nexp)}\}$, the square matrix $K \in \Re^{\nexp \times \nexp}$ whose entries
are given by $K_{ij} = K(x^{(i)},x^{(j)})$ is symmetric and positive
semidefinite. You can find more details about Mercer's theorem in the notes,
though the description above is sufficient for this problem.

Now here comes the question: Let $K_1$, $K_2$ be kernels over $\Re^{\di} \times
\Re^{\di}$, let $a \in \Re^+$ be a positive real number, let $f : \Re^{\di} \mapsto
\Re$ be a real-valued function, let $\phi: \Re^{\di} \rightarrow \Re^\nf$ be a
function mapping from $\Re^{\di}$ to $\Re^\nf$, let $K_3$ be a kernel over $\Re^\nf
\times \Re^\nf$, and let $p(x)$ a polynomial over $x$ with \emph{positive}
coefficients.

For each of the functions $K$ below, state whether it is necessarily a
kernel.  If you think it is, prove it; if you think it isn't, give a
counter-example.

\begin{enumerate}

\item \subquestionpoints{1} $K(x,z) = K_1(x,z) + K_2(x,z)$
\item \subquestionpoints{1} $K(x,z) = K_1(x,z) - K_2(x,z)$
\item \subquestionpoints{1} $K(x,z) = a K_1(x,z)$
\item \subquestionpoints{1} $K(x,z) = -a K_1(x,z)$
\item \subquestionpoints{5} $K(x,z) = K_1(x,z)K_2(x,z)$
\item \subquestionpoints{3} $K(x,z) = f(x)f(z)$
\item \subquestionpoints{3} $K(x,z) = K_3(\phi(x),\phi(z))$
\item \subquestionpoints{3} $K(x,z) = p(K_1(x,z))$

\end{enumerate}

[\textbf{Hint:} For part (e), the answer is that $K$ \emph{is} indeed
a kernel. You still have to prove it, though.  (This one may be harder than the
rest.)  This result may also be useful for another part of the problem.]

\ifnum\solutions=1 {
  \begin{answer}
\begin{enumerate}

\item
\subitem  (Symm)$K(x,z) = K_1(x,z) + K_2(x,z) = K_1(z,x) + K_2(z,x) = K(z,x)$
\subitem (PSD) $x^TK_1x + x^TK_2x = x^T(K_1 + K_2)x \ge 0$ 
\item (not PSD) $
x^T
\begin{bmatrix}
    1  & 0 \\
    0  & 1
\end{bmatrix} x - 
x^T
\begin{bmatrix}
    2  & 0 \\
    0  & 2
\end{bmatrix} x =
x^T
\begin{bmatrix}
    -1  & 0 \\
    0  & -1
\end{bmatrix} x =
- (x_1^2 + x_2^2) \le 0
$
\item 
\subitem  (Symm)$K(x,z) = a K_1(x,z) = a K_1(z,x) = K(z,x)$
\subitem (PSD) $x^TK(x,z)x = a x^TK_1(x,z)x \ge 0$
\item (not PSD) 
$
\begin{bmatrix}
    1  & 0 \\
    0  & 1
\end{bmatrix} 
$ - kernel
$
\begin{bmatrix}
    -1  & 0 \\
    0  & -1
\end{bmatrix} 
$ - not a kernel
\item
\subitem  (Symm)$K(x,z) = K_1(x,z)K_2(x,z) =  K_1(z,x)K_2(z,x)= K(z,x)$
\subitem (PSD) $K(x^{(i)}, x^{(j)}) =K_1(x^{(i)}, x^{(j)})K_2(x^{(i)}, x^{(j)}) = (\sum_{p, q} \phi_1^{(p)}(x^{(i)}) \phi_1^{(q)}(x^{(j)}))  (\sum_{p, q} \phi_2^{(p)}(x^{(i)}) \phi_2^{(q)}(x^{(j)})) =$
$\sum_{p, q} \phi_1^{(p)}(x^{(i)}) \phi_1^{(q)}(x^{(i)}) \phi_2^{(p)}(x^{(i)}) \phi_2^{(q)}(x^{(i)})$ 
$x^TKx = \sum_{p, q}  \sum_{i, j} x^{(i)} \phi_1^{(p)}(x^{(i)}) \phi_1^{(q)}(x^{(i)}) \phi_2^{(p)}(x^{(i)}) \phi_2^{(q)}(x^{(i)}) x^{(j)} =$
$\sum_{p, q} (\sum_{i} x^{(i)} \phi_1^{(p)}(x^{(i)}) \phi_2^{(q)}(x^{(i)}))^2$
\item 
\subitem  (Symm)$K(x,z) = f(x) f(z) = f(z) f(x) = K(z,x)$
\subitem (PSD) $x^TKx = \sum_{i, j} x^{(i)} f(x^{(i)}) x^{(j)} f(x^{(j)}) = (\sum_{i}x^{(i)} f(x^{(i)}) )^2 \ge 0$
\item
\subitem  (Symm) $K(x,z) = K_3(\phi(x), \phi(z)) = K_3(\phi(z), \phi(x)) = K(z,x)$
\subitem (PSD) $x^TKx = \sum_{i, j} x^{(i)} f(\phi(x^{(i)})) x^{(j)} f(\phi(x^{(j)})) = (\sum_{i}x^{(i)} f(\phi(x^{(i)})) )^2 \ge 0$
\item 
\subitem  (Symm) $K(x,z) = p(K_1(x, z)) = p(K_1(z, x))= K(z,x)$
\subitem (PSD) $
K = 
a_0\begin{bmatrix}
    1 & \dots  & 1 \\
    \vdots & \ddots & \vdots \\
    1 & \dots  & 1
\end{bmatrix} + 
a_1\begin{bmatrix}
    K_1(x^{(1)},x^{(1)}) & \dots  & K_1(x^{(1)},x^{(n)})  \\
    \vdots & \ddots & \vdots \\
    K_1(x^{(n)},x^{(1)})  & \dots  & K_1(x^{(n)},x^{(n()}) 
\end{bmatrix} + $ \\
$
a_2\begin{bmatrix}
    K_1(x^{(1)},x^{(1)})^2 & \dots  & K_1(x^{(1)},x^{(n)})^2  \\
    \vdots & \ddots & \vdots \\
    K_1(x^{(n)},x^{(1)})^2  & \dots  & K_1(x^{(n)},x^{(n()})^2
\end{bmatrix}   + ...
$
\subitem 0. $x^Ta_0\begin{bmatrix}
    1 & \dots  & 1 \\
    \vdots & \ddots & \vdots \\
    1 & \dots  & 1
\end{bmatrix}x \ge 0 $
\subitem 1. $a_1 x^T K_1 x \ge 0$
\subitem 2. $a_1 x^T K_1 \circ K_1 x \ge 0$ (consequence from (e))
\subitem 3. ...
\end{enumerate}

\end{answer}

} \fi
